% This is a LaTeX template for poster using University of Helsinki style
% This version uses pdflatex; if you have XeTeX available, 
% consider using poster_xetex.tex for fancy font settings
%
% The official Graphical Instructions available from hy.logodomain.com (from university network only) defines colours, fonts etc. This Template does not perfectly match the official poster template of the university
%
% Jukka Suomela's collection of LaTeX tricks has been invaluable in building this template: http://cs.helsinki.fi/u/josuomel/latex/ 
%
% Original template by Janne Korhonen
% This file by Juha Karkkainen

% Encoding of this file is iso-8859-1 latin 1

\documentclass[a4paper]{article} % This actually makes poster of size A4, scale up when printing. Used text font size is a bit small for A1 poster, preferably use \small in that case 

% Misc. packages
\usepackage[T1]{fontenc}
\usepackage{url}
\usepackage{amsfonts}
\usepackage[absolute]{textpos}
\usepackage{amssymb}
\usepackage{amsmath} 
\usepackage{amsthm}

% Graphics stuff
\usepackage[usenames,dvipsnames]{color}
\usepackage{graphicx}


% My favourite macros
\newcommand{\N}{\mathbb{N}} %natural numbers
\newcommand{\Z}{\mathbb{Z}} %integers
\newcommand{\Q}{\mathbb{Q}} %rationals
\newcommand{\R}{\mathbb{R}} %reals

\newcommand{\G}{\mathcal{G}} % fancy G
\newcommand{\A}{\mathcal{A}} % fancy A
\newcommand{\bO}{\mathcal{O}} % fancy O

\newcommand{\eos}{\#}
\newcommand{\rank}{\textsc{rank}}


% enumitem for controlling enumerate and itemize environments; usefull for saving space
\usepackage{enumitem}


% Fonts
%
% Palantino - Helvetica - Courier.
% I haven't really spent time to figure out how to best match the official university style with latex font packages, as I use XeTex myself...
\usepackage{mathpazo}
\linespread{1.10}
\usepackage[scaled]{helvet}
\usepackage{courier}

% Colours
\definecolor{sciorange}{RGB}{252,163,17}
\definecolor{unigray}{RGB}{140,140,140}
\definecolor{first}{RGB}{180,30,10}
\definecolor{second}{RGB}{10,120,130}
\definecolor{new}{RGB}{190,50,10}
\definecolor{wavelet}{RGB}{70,140,220}
%\definecolor{improved}{RGB}{80,200,80}
\definecolor{improved}{RGB}{10,100,160}
\definecolor{prior}{RGB}{140,140,140}

% Textpos to manually position blocks of text on the page
\usepackage[absolute]{textpos}

% We define 1 mm grid for positioning the text blocks on the page
% The idea is to leave 10 mm marginals to all sides; the three text columns are 60 mm wide with 5 mm space between columns.
\setlength{\TPHorizModule}{1mm}
\setlength{\TPVertModule}{1mm}

% The origin is set to right below the main title; this means that main title blocks have negative y-coordinate. There is actually no good reason for this, I just happened to do this that way.
\textblockorigin{10mm}{48.5mm}


% parindent is set to zero, because it looks better in posters
% you could also add some space between paragraphs here, but I use manual vertical spaces in this sample
\setlength{\parindent}{0pt}


% Finally, the content
\begin{document}
\pagestyle{empty} % To get rid of page numbers and so


% ------------------------------------------------------------------------------------------------------------------------------
% Main title, University logo etc.

% If you need more space for the title or want the logo to be bigger, you need to adjust various parameters, as I did not bother to automate this yet
% Mainly, move the textblockorigin above down and adjust all the boxes here to be higher up.

% First, the University logo. The included flame.pdf is a copy of the official logo in vector format, so it is good for all sizes.
% Box starts 10 mm from the top
\begin{textblock}{95}(0,-38.5)
\includegraphics[width=22mm]{flame}
\end{textblock}	

% University - Faculty - Department
% Box starts 10 mm from the top
\begin{textblock}{95}(95,-38.5)
{\fontsize{8}{7}\selectfont\sffamily\color{unigray}
\hfill HELSINGIN YLIOPISTO

\hfill HELSINGFORS UNIVERSITET

\hfill UNIVERSITY OF HELSINKI

\color{sciorange}\hfill MATEMAATTIS-LUONNONTIETEELLINEN TIEDEKUNTA

\hfill MATEMATISK-NATURVETENSKAPLIGA FAKULTETEN

\hfill FACULTY OF SCIENCE % For some reason, the last line here gets messed up without extra spaces...


}
\end{textblock}


% Main title
% If you need two lines for the title, you need to adjust the position of this block and textblockorigin
% Remember, two first words use faculty colour. If your title is long, you can use small letters.
\begin{textblock}{190}(0,-13.5)
{\sffamily\LARGE{{\color{sciorange}AHO-CORASICK AND RABIN-KARP\color{unigray} IN MULTIPLE PATTERN MATCHING}}}
\small\hfill Aleksi Hartikainen and Jussi Kokkala\\ % If you have multiple authors, their names can be on the same line. Adjust font size as necessary
\rule[2mm]{190mm}{0.3pt} % This is the line under the title, adjust the last parameter if it seems to be in the wrong place
\end{textblock}

% ------------------------- ABSTRACT ----------------------------------
% The "abstract" block. Does not actually exist in the university poster template, so you may consider not using this
\begin{textblock}{92.5}(0,0)
\sffamily
\small 
The Burrows-Wheeler transform (BWT) is a powerful tool for data
compression used for example in the popular compression program
bzip2. The \emph{inverse} BWT is usually the bottleneck in the
decompression phase with respect to both space and time.
\end{textblock}
\begin{textblock}{92.5}(97.5,0) 
  \sffamily \small 
  Our new algorithms improve the performance of inverse BWT.  They
  range from the fastest known algorithm to the most space-efficient
  one, and cover the whole space-time tradeoff spectrum in between.
\end{textblock}

 %Large subtitle with line. Again, not from the university template.
\begin{textblock}{190}(0,25)
\sffamily
\Large{\color{sciorange}ALGORITHMS}\small\\
\rule[3mm]{190mm}{0.1pt}
\end{textblock} 

% ---------------------------- INTRO ----------------------------------
% Three-column stuff
%
% The vertical size of the columns depends on the content, so unfortunately you have to manually move contents around

% First column
\begin{textblock}{60}(0,32)
  {\sffamily\normalsize{\color{sciorange}AHO--CORASICK
      }}\vspace{1mm}\\ % Titles among the main text are made like this, not by using \section


\end{textblock} 

% Second column
\begin{textblock}{60}(65,32)
  {\sffamily\normalsize{\color{sciorange}RABIN--KARP}}\vspace{1mm}\\
  \footnotesize 
      Rabin-Karp uses hashing to find fixed length patterns in a text. The hashing function used is a rolling hash, meaning that the hash of $T[2\;..\;N+1]$ can be calculated from the hash of $T[1..N]$ in constant time. 
     
If the hash values of $T[i\;..\;N+i]$ and a pattern prefix of length N, the algorithm checks if the pattern occurs at text index $i$.

    For $P$ patterns of combined length $m$  and a text of length $n$, the average running time of Rabin-Karp is $O(n+m)$ in space $O(P)$.
\end{textblock}


% ----------------------- EXPERIMENTS ----------------------------

\begin{textblock}{190}(0,97)
\sffamily
\Large{\color{sciorange}EXPERIMENTAL RESULTS}\small\\
\rule[3mm]{190mm}{0.1pt}
\end{textblock} 


\begin{textblock}{60}(0,103) 
  \footnotesize 
  The basic inversion algorithm described above has linear time and
  space complexity, but it still dominates the time and space
  requirements during decompression in programs like bzip2.
  \vspace{1mm}

  It is slow because each memory access during the permutation 
  traversal is essentially random causing many cache misses.
  \vspace{1mm}

  It needs a lot of space for the \rank\ array:
  \begin{align*}
    |\rank| &= n\log n \textrm{ bits} = 4n \textrm{ bytes}\\[-1mm]
    |\textrm{text}| &= n\log\sigma \textrm{ bits} = n \textrm{ bytes}
  \end{align*}
  where $n={}$text length and $\sigma={}$alphabet size.
\end{textblock}

\begin{textblock}{60}(0,152)
  {\sffamily\normalsize{\color{sciorange}
      REFERENCE POINT RANKS}}\vspace{1mm}\\
  \footnotesize 
  We reduce space by storing ranks relative to reference points,
  which can be placed in two ways:\\

  \begin{minipage}[t]{25mm}
    \scriptsize\sffamily
    \centering
    Every $k$th position~\cite{ll2005}
    \begin{center}
      
    \end{center}
  \end{minipage}
  \hfill
  \begin{minipage}[t]{30mm}
    \scriptsize\sffamily
    \centering
    Every $k$th occurrence [new]
    \begin{center}
      
    \end{center}
  \end{minipage}
  \vspace{3mm}

  \begin{minipage}{42mm}
%\scriptsize\sffamily
%\centering
    \raggedright
    A new improvement is to use variable length encoding, where
    a frequent symbol uses less bits for symbol and more bits for rank.
  \end{minipage}
  \hfill
  \begin{minipage}{17mm}
    \begin{center}
      
    \end{center}
  \end{minipage}
  \vspace{1mm}

  Finally, we can trade time for space by replacing \rank\ with
  scanning from the nearest reference point~\cite{ll2005}.
\end{textblock}

\begin{textblock}{60}(65,104) 
  {\sffamily\normalsize{\color{sciorange}
      REPETITION SHORTCUTS}}\vspace{1mm}\\
  \footnotesize 
  Repetitions in the text manifest as \emph{pairs of parallel paths}
  (PPP) in the inverse BWT permutation. We use this as follows.
  \vspace{3mm}

  \begin{minipage}[t]{29mm}
    \scriptsize\sffamily
    \centering
    1. On the {\color{first}first pass},\\observe the PPP\\
    (due to repeated ANA)
    \begin{center}
      
    \end{center}
  \end{minipage}
  \hfill
  \begin{minipage}[t]{29mm}
    \scriptsize\sffamily
    \centering
    2. Replace the {\color{second}other path} by shortcut and
    follow it\\on the second pass
    \begin{center}
      
    \end{center}
  \end{minipage}
  \vspace{3mm}
  
  The shortcuts reduce the number of cache misses.  This is the
  \emph{fastest} known algorithm.
\end{textblock} 

\begin{textblock}{60}(65,173)
  {\sffamily\normalsize{\color{sciorange}
      WAVELET TREES}}\vspace{1mm}\\
  \footnotesize 
  \emph{Wavelet tree} is a text representation that can be both
  compressed and preprocessed for rank que\-ries with little
  additional space.  They are used with compressed text
  indexes~\cite{nm2007} to answer \emph{general rank queries}:
  \[
  \rank_c(j)=\big|\{i \mid i<j \textrm{ and }
  L[i]=c\}\big|.
  \]
  We need wavelet trees for \emph{special rank queries}:
  \[
  \rank(j)=\rank_{L[j]}(j).
  \]
  We use our own wavelet tree implementations optimized for special
  rank queries.
  \vspace{1mm}

  We combine wavelet trees with reference point
  ranks, obtaining the \emph{most space-efficient} algorithm.
\end{textblock} 

%\begin{textblock}{60}(130,165)
%  {\sffamily\normalsize{\color{sciorange}CONCLUDING REMARKS}}\vspace{1mm}\\
%  \footnotesize 
%\end{textblock}

\begin{textblock}{60}(130,185)
  \def\refname{\normalfont\sffamily\normalsize{\color{sciorange}REFERENCES}}
  \scriptsize\sffamily
  \bibliographystyle{abbrv}
  \bibliography{bib}
\end{textblock}

% Unfortunately, this template has no references. The official instructions seem to indicate that sans-serif font is used for the reference list.

\end{document}
